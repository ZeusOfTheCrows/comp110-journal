% Please do not change the document class
\documentclass{scrartcl}

% Please do not change these packages
\usepackage[hidelinks]{hyperref}
\usepackage[none]{hyphenat}
\usepackage{setspace}
\doublespace

% You may add additional packages here
\usepackage{amsmath}

% Please include a clear, concise, and descriptive title
\title{The influence of “Computing Machinery and Intelligence”}

% Please put your student number in the author field 
\author{1807684}

\begin{document}

\maketitle

\section{Introduction}
\paragraph{}
The phrase “Turing Test” is ubiquitous in modern society; even if many people do not fully understand the stipulations of the original idea, most still know the general gist.

\section{The Education Proposal}
\paragraph{}  % what am I doing with my life
Turing suggested\cite{Main} that rather than attempting to create an entire fully-fledged consciousness from scratch, a simpler—and perhaps more reliable—method would be to create a much more basic "infant" AI, and expose it to a variety of environments in order for it to learn. This expands the problem to something much more easily grasped by a non-technical mind.\cite{Infants}


\section{The Fallible Mind}
\paragraph{}
In order to be indistinguishable from a human mind, a computer must not  only be able to successfully imitate the mind's more artistic and emotional capabilities, such as composing a moving piece of music, it must also display the mind's weaknesses\cite{AIProgress}. This means that it must give a varying delay before providing answers; as 
%Find me - rephrase this
well as serving answers that are convincingly inaccurate (e.g., Forgetting a person's surname, yet remembering the first letter). This means that a machine that could pass the imitation game, whilst being impressive imitation of human intelligence, isn't that useful in a practical setting\cite{Counterpoint}.

\section{Conclusion}
\paragraph{}


\bibliographystyle{ieeetr}
\bibliography{refs}

\end{document}
